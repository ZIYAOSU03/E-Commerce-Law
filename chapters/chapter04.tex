\section{完善路径:以“可得性、可核验、可追溯”为中心的制度组合}

\subsection{确立可执行的分层公示标准}
建议将信息公示分为“基础必备信息”与“特定业态补充信息”两层:基础层统一要求经营者在所有端口(网页、APP、小程序、直播间/带货入口)提供一键可达的主体身份与联系方式;补充层则针对食品药品、教育培训、跨境电商等高风险领域增加许可、资质、投诉渠道等要素。通过分层标准既能避免“一刀切”造成合规负担过重,也能提升高风险领域的消费者保护强度。

\subsection{引入强制可核验机制:从“贴图”走向“可校验标识”}
仅展示图片难以防伪。可借鉴“可点击校验”的思路:平台侧在经营者完成资质提交与核验后,生成统一的可核验标识(如可点击跳转到平台核验页,展示核验时间、有效期、许可范围等)。监管部门可推动平台与登记系统/许可系统对接,实现自动比对与到期提醒。这样既降低消费者核验成本,也使平台治理与行政监管形成数据联动。

\subsection{强化动态更新与变更提示:让信息披露贯穿交易全流程}
建议对信息变更设置更明确的“同步义务”与“提示义务”:经营者变更名称、地址、许可范围等关键信息时,除更新公示页外,应在一定期间内对已下单消费者提供可见提示(如订单页提示、站内信提示),并确保历史订单可追溯到当时有效信息。对频繁变更且投诉率高的主体,平台可实施重点审查或限制经营措施。

\subsection{平台协同治理:把入口责任转化为可监督的流程责任}
平台掌握流量入口与交易链路,是实现“信息可得”的关键节点。建议在平台规则中明确:未完成基础信息公示与核验的经营者不得开店或不得上架商品;对直播带货等新业态,应确保主体信息在直播间界面持续可见。同时引入外部可审计机制,例如定期披露平台抽检比例、违规处置数据,并接受监管抽查或第三方评估,避免平台治理沦为“内部口径”。

\subsection{责任与惩戒衔接:让违法成本高于逃逸收益}
对拒不公示、虚假公示、长期不更新等行为,应在行政处罚、信用惩戒与平台处置之间形成合力:行政机关依法处罚并纳入信用记录;平台实施限流、下架、关店等处置;对造成严重后果的,可依法追究民事赔偿乃至刑事责任。通过提高综合违法成本,改变不诚信主体的成本收益结构,使信息公示从“可做可不做”变为“必须做好”。



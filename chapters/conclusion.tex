\section{结语}
信息公示义务的关键,不是“有没有展示”,而是“能不能被有效使用”。如果入口隐蔽、信息不可核验、变更不同步,消费者仍会陷入识别与追责困境。本文以“信息可得”为解释轴心,提出以可得性(显著入口)、可核验(点击校验)、可追溯(历史留痕)为核心的改进方向,并强调平台协同与责任闭环。通过把规则转化为可执行流程,才能让“亮照亮证”真正服务于交易安全与消费者保护。

同时也应看到,信息披露并非越多越好:过度披露可能造成信息过载,反而削弱消费者的理解与选择。未来完善制度时,应在“关键事实充分披露”与“用户可理解”之间取得平衡,并结合不同业态风险水平动态调整披露与核验强度,使规则既能落地又不损害交易效率。



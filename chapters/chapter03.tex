\section{完善路径:以“可得性、可核验、可追溯”为核心的可执行方案}

\subsection{做实“可得性”:统一入口与分层披露}
首先要解决“找不到”的问题:在网页、APP、小程序、直播间等端口统一提供“一键可达”的主体信息入口;将主体名称、联系方式、许可资质、售后规则等关键信息置于优先位置。对高风险行业适当增加补充披露要素,实现分层治理。

\subsection{增强“可核验”:从贴图到可点击校验}
仅展示图片难以防伪。可行做法是引入“可点击校验页”:平台核验后生成统一标识,消费者点击即可看到核验时间、有效期与许可范围;同时设置到期提醒与抽检机制,使真实性可检查。

\subsection{补齐“可追溯”:变更同步+历史留痕}
信息公示必须贯穿交易全流程。建议明确两点:其一,关键信息变更应及时同步到公示页;其二,历史订单应能回溯到“下单当时”的有效信息(含主体、规则与联系方式),并允许消费者留存。对频繁变更且投诉率高的经营者,平台应触发强化审查、限制上新或限制结算等措施,抑制“动态逃逸”收益。

\subsection{平台协同与责任闭环:让违法成本高于逃逸收益}
最后形成“闭环”:平台把“未完成基础公示/核验不得经营”规则化并稳定执行;对拒不公示、虚假公示、长期不更新等行为,与行政监管和信用惩戒联动,提高综合违法成本,压缩逃逸收益。



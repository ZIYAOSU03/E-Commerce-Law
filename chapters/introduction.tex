\section{引言}

\subsection{选题背景与问题意识}
电子商务交易具有非接触、跨地域与强平台化特征,消费者对经营者的判断高度依赖页面信息。若经营者隐匿主体或频繁变更信息,消费者就会遭遇识别困难与维权成本上升。信息公示义务意在固定“谁在卖、如何联系、能否追责”等关键信息;但实践中常被简化为贴图或编号,出现形式合规与实质效果的落差。

\subsection{研究思路与结构安排}
本文采用规范分析与问题导向相结合的方法:第一章阐释规范要点与价值;第二章总结实践问题与成因;第三章提出以可得性、可核验、可追溯为核心的改进方案;结语归纳观点。



\section{引言}

\subsection{选题背景与问题意识}
电子商务场景中,交易双方往往不见面、不签纸质合同、交易链条跨平台跨地域,消费者对经营者的了解高度依赖页面信息。与此同时,经营者“低成本进入—快速退出”的特征,使得虚假主体、资质不明、售后失联等风险更易发生。与传统线下交易相比,电子商务的效率优势在相当程度上来自信息与流程的数字化,但这一优势也可能被不诚信主体利用,形成对消费者不利的结构性信息差。

基于此,《中华人民共和国电子商务法》通过确立“经营者身份可识别”的基本要求,设置了以登记、公示、持续披露为核心的信息公示义务。该义务一方面为消费者提供最低限度的识别与追责线索,另一方面也为监管部门开展穿透式治理提供抓手。但在实践中,信息公示常被异化为“挂在角落的一张图片”或“看不懂、点不开的备案号”,出现展示不显著、信息不真实、变更不同步、跨端不一致、核验困难等问题,从而削弱了制度的实际效果。

\subsection{研究思路与结构安排}
本文选择“电子商务经营者信息公示义务”作为讨论对象,力图回答三个问题:第一,信息公示义务的规范内涵应如何理解,如何区分“形式合规”与“实质可得”;第二,该义务在消费者保护、市场秩序与监管效率上的制度价值何在;第三,针对实践中的落差,应当如何在经营者自律、平台协同与监管执法之间形成可执行的完善路径。

全文结构如下:第一章界定信息公示义务的规范基础与内涵边界;第二章从交易安全与治理逻辑角度阐释该义务的制度价值;第三章梳理实践痛点并分析成因;第四章提出以“可得性、可核验、可追溯”为导向的改进建议;结语部分总结观点并提出后续研究方向。



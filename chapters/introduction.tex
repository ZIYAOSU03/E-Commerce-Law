\section{引言}

\subsection{选题背景与问题意识}
电子商务交易具有非接触、跨地域与强平台化特征。对消费者而言,线下交易中“看店面、看招牌、看营业执照”的直观判断被压缩为手机屏幕上的几行文字、几张图片与一串链接。许多纠纷的起点并不复杂:例如直播间里一款“同款低价”商品迅速成交,收货后发现与宣传不符,消费者试图联系商家,却只能在多级页面里找到一个不常用的客服入口;或是店铺信息显示“个体户”,但实际履约与收款主体又指向另一公司,投诉时平台与经营者之间责任边界模糊,消费者只能在不同主体之间来回奔走。

正因为“识别—联系—追责”是网络交易中最基础的安全链条,《电子商务法》把信息公示义务放在制度底座:要求经营者依法登记,并在首页显著位置持续披露营业执照、许可资质以及与消费者权益密切相关的规则信息。\footnote{《中华人民共和国电子商务法》(2018年8月31日通过,2019年1月1日起施行)。} 但在实践中,这一义务常被简化为“贴一张执照图”“挂一个备案号”,入口不显著、信息不可核验、变更不同步等问题仍普遍存在,导致形式合规与实质效果之间出现落差。本文的核心问题因此可以表述为:信息公示义务究竟要解决什么风险?“亮照亮证”如何从形式展示走向可得、可信、可用的消费者保护能力?

\subsection{研究思路与结构安排}
本文采用规范分析与问题导向相结合的方法:第一章在“信息可得”的视角下解释信息公示义务的规范要点与制度价值,并通过要素表梳理经营者应当披露的信息清单;第二章从平台化交易链条出发,总结实践中的典型困境并提出可执行的治理路径(包括统一入口、可点击核验、变更留痕与责任闭环),并用流程图与示意统计图呈现治理重点;结语部分回扣本文观点并讨论信息披露与用户可理解之间的平衡。



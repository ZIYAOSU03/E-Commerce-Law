\section{实践痛点与成因:形式合规为何难以转化为实质效果}

\subsection{入口不显著与跨端不一致:消费者“找不到”}
实践中最常见的问题是入口被折叠:网页端似乎“有公示”,但在 APP、小程序、直播间或分享短链里难以找到。对消费者而言,“找不到”在效果上等同于未公示,也容易造成取证困难。

\subsection{真实性与可核验不足:贴图不等于可信}
执照贴图可能被裁剪、过期或伪造;若缺乏便捷核验渠道,公示信息就会沦为“装饰性合规”,难以发挥风险预防作用。

\subsection{变更不同步与动态逃逸:追责对象被不断“换皮”}
网络经营的流动性强,主体名称、地址、许可范围甚至店铺经营主体都可能频繁变更。若变更后仍沿用旧信息,消费者就会遭遇追责错位;更极端的情形是通过多主体轮换、店铺转让等方式实现“动态逃逸”。相关规制已明确网络交易治理要求并强化平台与经营者的合规责任,但落地仍依赖技术与流程。\footnote{《网络交易监督管理办法》(国家市场监督管理总局令,2021年施行)。}

\subsection{成因:激励失衡与平台治理标准不统一}
成因具有结构性:不诚信主体有隐藏信息的动机;平台治理若缺乏统一标准与外部监督,容易“选择性治理”。关键在于把规则固化为可验证、可执行的流程。



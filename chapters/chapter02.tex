\section{实践困境与完善路径:从“形式合规”走向“信息可得”}

\subsection{入口不显著与跨端不一致:消费者“找不到”}
实践中最常见的问题是入口被折叠:网页端似乎“有公示”,但在 APP、小程序、直播间或分享短链里难以找到。对消费者而言,“找不到”在效果上等同于未公示,也容易造成取证困难。

\subsection{真实性与可核验不足:贴图不等于可信}
执照贴图可能被裁剪、过期或伪造;若缺乏便捷核验渠道,公示信息就会沦为“装饰性合规”,难以发挥风险预防作用。

\subsection{变更不同步与动态逃逸:追责对象被不断“换皮”}
网络经营的流动性强,主体名称、地址、许可范围甚至店铺经营主体都可能频繁变更。若变更后仍沿用旧信息,消费者就会遭遇追责错位;更极端的情形是通过多主体轮换、店铺转让等方式实现“动态逃逸”。相关规制已明确网络交易治理要求并强化平台与经营者的合规责任,但落地仍依赖技术与流程。\footnote{《网络交易监督管理办法》(国家市场监督管理总局令,2021年施行)。}

\subsection{成因:激励失衡与平台治理标准不统一}
成因具有结构性:不诚信主体有隐藏信息的动机;平台治理若缺乏统一标准与外部监督,容易“选择性治理”。关键在于把规则固化为可验证、可执行的流程。

\subsection{完善路径:以可得性、可核验、可追溯为主线的连续治理}
要让信息公示义务真正落地,不能只停留在“要求展示”,而应当围绕消费者体验把义务做成可检查的流程:能否快速找到、是否真实可核验、变更后能否追溯到下单时的有效信息。由此,完善路径可概括为“可得性—可核验—可追溯”的连续治理,并通过平台规则与监管联动保证执行力。

在\textbf{可得性}层面,应把“首页显著位置”具体化为跨端一致的统一入口与较短的获取路径,避免信息只在网页端可见、在 APP/小程序/直播间被隐藏;平台可将入口可达性纳入准入与抽检。 在\textbf{可核验}层面,应减少单纯贴图,推广可点击校验页与统一标识,展示核验时间、有效期与许可范围,并配合到期提醒与抽检,压缩伪造与过期信息的生存空间。 在\textbf{可追溯}层面,应要求变更同步与历史留痕,使历史订单可回溯到下单时的主体与规则信息,减少追责错位。最后以\textbf{责任闭环}收束:平台稳定执行“未完成基础公示/核验不得经营”,对拒不公示、虚假公示、长期不更新等行为与行政监管、信用惩戒联动,提高违法成本,使隐藏信息不再成为理性选择。



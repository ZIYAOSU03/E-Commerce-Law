\section{信息公示义务的制度价值:消费者保护与市场治理的结合}

\subsection{降低信息不对称:把“信任成本”转化为可计算的风险}
网络交易中,消费者对经营者的了解主要来源于页面描述、评价体系与平台推荐机制。若缺乏身份与资质的最低披露,消费者只能在高度不确定中作出选择,信任成本上升,并最终表现为“劣币驱逐良币”:诚信经营者因合规成本与售后投入更高而处于劣势,不诚信经营者则通过低价与短周期获利后退出市场。

信息公示义务将经营者最关键的主体与资质信息制度化,使消费者能够在交易前进行基本风险判断,减少“买到假货才发现找不到人”的事后救济困境。其底层逻辑可用“信息不对称导致市场失灵”的经典观点来解释\cite{InformationAsymmetry1970}:当优质与劣质难以区分时,市场会系统性地惩罚诚信与质量。信息公示通过降低识别成本,帮助交易回到可竞争的轨道。这不仅服务于个体消费者,更能通过提升整体交易确定性,降低社会总交易成本。

\subsection{强化可追责性:为救济机制提供“抓手”}
消费者权益保护的难点常不在“权利是否存在”,而在“责任主体是否可确定”。在电子商务纠纷中,经营者失联、地址虚假、主体频繁变更等现象,会使投诉、调解、诉讼乃至强制执行陷入空转。信息公示义务通过“经营者身份可识别”这一底层设计,为后续的合同责任、侵权责任与行政责任衔接提供基础要素。

进一步而言,信息公示还具有证据功能:消费者截图留存的主体信息、规则信息与联系方式,在争议解决中往往构成关键证据。若经营者被允许以“隐藏信息”方式经营,则实际上变相剥夺了消费者的证据准备能力。

\subsection{提升监管效率:从事后处罚走向事前预防}
对监管机关而言,互联网交易规模巨大,单纯依赖事后处罚难以覆盖全量风险。信息公示义务为监管提供了“低成本筛查”的入口:通过公示信息的完整性、真实性与更新及时性,可以对高风险主体开展重点监管;通过平台侧的技术核验与黑名单共享,可以实现早发现、早处置。

因此,信息公示义务兼具私法与公法双重价值:在私法层面,它保护交易相对方的知情与选择;在公法层面,它服务于市场秩序与监管可达性。其关键不在“是否张贴”,而在“是否能被利用来形成有效治理”。



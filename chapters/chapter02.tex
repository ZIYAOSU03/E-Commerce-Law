\section{实践困境与完善路径:从“形式合规”走向“信息可得”}

\subsection{入口不显著与跨端不一致:消费者“找不到”}
实践中最常见的问题是入口被折叠:网页端似乎“有公示”,但在 APP、小程序、直播间或分享短链里难以找到。更麻烦的是同一店铺在不同端展示的主体信息并不一致——消费者截图留存的页面信息,与后台登记信息存在差异时,争议解决会立刻变得复杂。对普通消费者而言,“找不到”在效果上等同于未公示;而跨端不一致则意味着证据链更容易断裂。部门规章层面对“网络交易主体信息”“平台治理责任”等也提出了更细化的要求,意在把这种“入口与一致性”问题从软要求变成可执行的合规标准。\footnote{《网络交易监督管理办法》(国家市场监督管理总局令,2021年施行)。}

\subsection{真实性与可核验不足:贴图不等于可信}
执照贴图看起来直观,但其真实性往往难以判断:图片可能被裁剪、模糊处理,也可能在主体变更后长期不更新,甚至存在伪造风险。对于消费者而言,缺乏便捷核验渠道意味着“知道有信息”却“无法确认信息是否可靠”;对于监管而言,在海量经营者面前依赖人工核查并不现实。于是公示信息很容易沦为“装饰性合规”,其风险预防功能被削弱。更进一步,电商页面往往与个人信息处理、账号体系绑定,当平台在投诉、核验、送达等环节调取或处理相关数据时,也必须遵守个人信息保护的合法、正当、必要原则。\footnote{《中华人民共和国个人信息保护法》。}

\subsection{变更不同步与动态逃逸:追责对象被不断“换皮”}
网络经营的流动性强,主体名称、地址、许可范围甚至店铺经营主体都可能频繁变更。若变更后仍沿用旧信息,消费者就会遭遇追责错位;更极端的情形是通过多主体轮换、店铺转让等方式实现“动态逃逸”。与线下相比,网络交易的“送达与通知”更依赖平台的技术与规则安排:如果经营者的联系方式形同虚设,消费者难以完成有效通知,争议解决成本就会迅速上升。\footnote{关于网络交易中证据与程序衔接,可参见《最高人民法院关于互联网法院审理案件若干问题的规定》(法释〔2018〕16号)。}

\subsection{成因:激励失衡与平台治理标准不统一}
成因具有结构性:不诚信主体有隐藏信息的动机;平台治理若缺乏统一标准与外部监督,容易“选择性治理”。关键在于把规则固化为可验证、可执行的流程。国际组织关于线上消费者保护的相关文件也强调,平台经济下需要通过透明度与可追责机制降低信息不对称,避免将风险外部化给消费者。\footnote{参见 OECD, \textit{OECD Recommendation on Consumer Protection in E-commerce}(2016),关于在线交易透明度、披露与救济的政策建议。}

\begin{figure}[H]
\centering
\caption{信息公示合规链条示意(从准入到争议解决)}
\label{fig:compliance-flow}
\begin{tikzpicture}[
  node distance=10mm and 12mm,
  box/.style={rectangle, rounded corners, draw=black, align=center, minimum width=32mm, minimum height=10mm, fill=gray!10},
  arrow/.style={-Latex, thick}
]
\node[box] (a) {入驻/开店\\提交主体与资质};
\node[box, right=of a] (b) {平台核验\\生成可校验标识};
\node[box, right=of b] (c) {多端统一入口\\显著公示};
\node[box, right=of c] (d) {信息变更\\同步+留痕};
\node[box, below=of c] (e) {纠纷发生\\截图取证/投诉};
\node[box, left=of e] (f) {责任对接\\联系/送达/追责};

\draw[arrow] (a) -- (b);
\draw[arrow] (b) -- (c);
\draw[arrow] (c) -- (d);
\draw[arrow] (c) -- (e);
\draw[arrow] (e) -- (f);
\draw[arrow] (f) |- (b);
\end{tikzpicture}
\end{figure}

\begin{figure}[H]
\centering
\caption{电商纠纷中“信息问题”相关成因分布(示意)}
\label{fig:bar-demo}
\begin{tikzpicture}[x=0.3cm,y=0.85cm]
% 说明:以下为示意性数据,用于课堂论文展示结构性问题,不代表官方统计。
\draw[->] (0,0) -- (26,0) node[right] {占比(\%)};
\draw[->] (0,0) -- (0,6) node[above] {类别};

% 注意:列表不要额外套一层 {},否则会被当成单个元素导致排版异常
\def\labels{入口难找,信息不一致,联系方式失效,资质难核验,变更不同步}
\def\values{22,18,20,16,24}

\foreach \lab [count=\n from 0] in \labels {
  \node[anchor=east] at (-0.6,5.4-\n) {\lab};
}
\foreach \val [count=\n from 0] in \values {
  \fill[gray!40] (0,5.15-\n) rectangle (\val,5.65-\n);
  \node[anchor=west] at (\val+0.4,5.4-\n) {\val};
}
\node[anchor=west] at (0, -0.9) {\footnotesize 注:示意性数据,基于公开案例类型归纳,用于说明结构性风险点。};
\end{tikzpicture}
\end{figure}

\subsection{完善路径:以可得性、可核验、可追溯为主线的连续治理}
要让信息公示义务真正落地,不能只停留在“要求展示”,而应当围绕消费者体验把义务做成可检查的流程:能否快速找到、是否真实可核验、变更后能否追溯到下单时的有效信息。由此,完善路径可概括为“可得性—可核验—可追溯”的连续治理,并通过平台规则与监管联动保证执行力。

在\textbf{可得性}层面,应把“首页显著位置”具体化为跨端一致的统一入口与较短的获取路径,避免信息只在网页端可见、在 APP/小程序/直播间被隐藏;平台可将入口可达性纳入准入与抽检。 在\textbf{可核验}层面,应减少单纯贴图,推广可点击校验页与统一标识,展示核验时间、有效期与许可范围,并配合到期提醒与抽检,压缩伪造与过期信息的生存空间。 在\textbf{可追溯}层面,应要求变更同步与历史留痕,使历史订单可回溯到下单时的主体与规则信息,减少追责错位。最后以\textbf{责任闭环}收束:平台稳定执行“未完成基础公示/核验不得经营”,对拒不公示、虚假公示、长期不更新等行为与行政监管、信用惩戒联动,提高违法成本,使隐藏信息不再成为理性选择。

进一步说,以上路径之所以可行,是因为它把“义务”变成了“流程”。当平台把准入核验、统一入口、变更留痕等环节固化下来,消费者在纠纷发生时就能自然地沿着流程寻找责任主体:先在统一入口确认主体与联系方式,再以截图与订单信息完成取证,随后将争议导向投诉、调解或诉讼。相反,如果平台在入口上留出太多“可钻的空子”,或对变更与留痕缺乏强约束,那么再完备的实体规则也会在执行端失去支点。对监管机关而言,流程化也意味着更低的执法成本:与其事后逐一核查,不如把抽检与到期提醒嵌入平台系统,让风险在早期被识别并处置。



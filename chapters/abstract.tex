\section*{摘要}
\addcontentsline{toc}{section}{摘要}

电子商务放大了交易中的信息不对称:消费者多依赖页面呈现,难以准确识别经营者身份与资质,纠纷中易出现“找不到人、追不了责”。为降低结构性风险,《电子商务法》确立经营者信息公示义务。本文主张将合规标准从“贴出证照”提升为“信息可得”,并围绕入口显著、内容真实、便于核验、变更同步提出改进建议,强调平台协同与责任闭环以提升制度实效。
电子商务放大了交易中的信息不对称:消费者往往在“看得见商品、看不清主体”的情境下完成下单,一旦发生质量争议或退款纠纷,才发现经营者联系方式不明、主体信息难以核验,维权成本随之上升。为降低这种结构性风险,《电子商务法》确立了经营者的信息公示义务,要求经营者在首页显著位置持续披露营业执照、许可资质与关键规则信息。本文结合常见交易场景,提出应当把合规标准从“贴出证照”提升为“信息可得”,并从可得性、可核验、可追溯三个维度阐释其制度价值与实践难点。文章进一步提出以统一入口、可点击核验、变更留痕与平台协同治理为抓手的改进路径,并辅以表格与示意图展示合规要点,力求使该义务真正转化为消费者可用的识别与救济能力。

\textbf{关键词:}电子商务法;信息公示义务;亮照亮证;消费者保护;平台治理



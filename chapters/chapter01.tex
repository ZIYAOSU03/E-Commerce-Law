\section{信息公示义务的规范要点与制度价值}

\subsection{规范要点:公示什么、如何公示、何时更新}
信息公示义务的核心不在“展示一张证照”,而在于确保经营者在网络交易中始终“可识别、可联系、可追责”。《中华人民共和国电子商务法》要求电子商务经营者依法登记,并在首页显著位置持续公示营业执照信息、行政许可信息等关键身份要素。\footnote{《中华人民共和国电子商务法》(2018年8月31日通过,2019年1月1日起施行)。}

从合规结构看,可以概括为“三个问题”:
\begin{enumerate}[label=(\arabic*)]
  \item \textbf{公示什么:}主体身份信息(名称、统一社会信用代码/登记信息、联系方式等)、许可资质信息(依法需许可的行业)、与消费者权益密切相关的规则信息(售后、争议处理等)。
  \item \textbf{如何公示:}入口易发现、少层级、跨端一致;表达应可理解,避免“贴图即完事”。
  \item \textbf{何时更新:}关键信息变更应及时同步,避免基于过期信息交易。
\end{enumerate}

\subsection{制度价值:把“信息差”转化为可管理的风险}
信息公示义务的重要性在于网络市场天然存在信息不对称:消费者难以区分“真实可追责”与“短期套利后退出”。相关研究指出,质量不确定会诱发市场失灵。\footnote{George A. Akerlof, The Market for ``Lemons'': Quality Uncertainty and the Market Mechanism, \textit{The Quarterly Journal of Economics}, 1970, 84(3): 488--500.} 信息公示通过降低识别与核验成本,提升交易确定性。

因此,本文主张以“信息可得”作为解释轴心,将合规评价从“有没有贴出来”推进到“能不能被消费者有效使用”。最低限度应满足:
\begin{enumerate}[label=(\arabic*)]
  \item \textbf{可见:}入口显著、少层级、跨端一致;
  \item \textbf{可信:}信息真实、可核验、不过期;
  \item \textbf{可用:}足以联系经营者、留存证据并支撑后续救济。
\end{enumerate}
换言之,“亮照亮证”只是形式起点,“信息可得”才是效果终点。



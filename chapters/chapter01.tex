\section{信息公示义务的规范要点与制度价值}

\subsection{规范要点:公示什么、如何公示、何时更新}
信息公示义务的核心不在“展示一张证照”,而在于确保经营者在网络交易中始终“可识别、可联系、可追责”。《中华人民共和国电子商务法》要求电子商务经营者依法登记,并在首页显著位置持续公示营业执照信息、行政许可信息等关键身份要素。\footnote{《中华人民共和国电子商务法》(2018年8月31日通过,2019年1月1日起施行)。}

从合规结构看,这一义务回答三个连续问题:\textbf{公示什么}(主体身份、必要资质与关键规则)、\textbf{如何公示}(入口显著、表达可理解、跨端一致)、\textbf{何时更新}(变更及时同步,避免依据过期信息交易)。

\subsection{制度价值:把“信息差”转化为可管理的风险}
信息公示义务的重要性在于网络市场天然存在信息不对称:消费者难以区分“真实可追责”与“短期套利后退出”。相关研究指出,质量不确定会诱发市场失灵。\footnote{George A. Akerlof, The Market for ``Lemons'': Quality Uncertainty and the Market Mechanism, \textit{The Quarterly Journal of Economics}, 1970, 84(3): 488--500.} 信息公示通过降低识别与核验成本,提升交易确定性。

因此,本文主张以“信息可得”作为解释轴心,将合规评价从“有没有贴出来”推进到“能不能被消费者有效使用”。所谓可得,至少包含三个层面的实质效果:其一是\textbf{可见},即入口显著、获取路径短、跨端一致;其二是\textbf{可信},即信息真实可核验、不过期不缺项;其三是\textbf{可用},即消费者能够据此联系经营者、留存证据并在争议解决中完成责任对接。换言之,“亮照亮证”只是形式起点,“信息可得”才是效果终点;法律要保障的并非页面上有一张图,而是消费者在交易全流程中都能抓住真实的责任主体。



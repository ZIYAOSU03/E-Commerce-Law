\section{信息公示义务的规范要点与制度价值}

\subsection{规范要点:公示什么、如何公示、何时更新}
信息公示义务常被概括为“亮照亮证”,但如果把它理解成“页面上放一张图片”,就很容易走到形式化的死胡同。本文更愿意把它理解为一种“可识别义务”:经营者在网络交易中必须保持可识别、可联系、可追责,使消费者能够以较低成本确认交易相对人并保存证据。\footnote{《中华人民共和国电子商务法》(2018年8月31日通过,2019年1月1日起施行)。另可参见联合国国际贸易法委员会(UNCITRAL)《电子商务示范法》(1996年),其以“功能等同”思路处理电子交易中的要件与可证明性问题。}

从合规结构看,这一义务回答三个连续问题:\textbf{公示什么}——主体身份信息(名称、登记信息、联系方式等)与必要的许可资质,以及影响消费者作出交易决策的关键规则(例如售后、争议处理与退款流程);\textbf{如何公示}——入口应显著、路径应短、表达应可理解,并在网页、APP、小程序、直播间等端口保持一致,避免把重要信息藏在多级跳转或冗长条款中;\textbf{何时更新}——“持续公示”意味着关键信息变更应及时同步,防止消费者依据过期信息作出判断,从而在事后救济中遭遇主体错位。与此相衔接的还有广告与宣传的真实性要求:当页面展示对消费者形成交易诱导时,经营者更应承担以真实信息支撑其宣传的责任。\footnote{《中华人民共和国广告法》。}

\begin{table}[H]
\centering
\caption{信息公示要素清单(基础项与强化项)}
\label{tab:disclosure-checklist}
\renewcommand{\arraystretch}{1.2}
\begin{tabularx}{\textwidth}{@{}p{2.8cm}X p{4.2cm}@{}}
\toprule
\textbf{披露层级} & \textbf{建议披露内容(举例)} & \textbf{对应风险(未披露/不真实)} \\
\midrule
基础项(所有经营者) & 主体名称、登记信息/统一社会信用代码、有效联系方式(电话/在线客服/地址或可送达地址)、营业执照信息入口 & 主体难以识别、无法联系、投诉与送达困难、追责成本上升 \\
基础项(与消费者权益相关) & 售后与退换规则、争议解决与投诉渠道、退款处理时限、格式条款提示(关键限制应醒目) & 决策信息缺失、规则不透明、争议中举证困难 \\
强化项(高风险业态/行业) & 许可资质与范围(食品/药品/教育培训等)、有效期、核验时间、客服响应机制、风险提示 & 无证经营风险、过期资质、诱导交易、合规审查难落地 \\
\bottomrule
\end{tabularx}
\end{table}

\subsection{制度价值:把“信息差”转化为可管理的风险}
信息公示义务的重要性在于网络市场天然存在信息不对称:消费者很难区分“真实可追责的长期经营者”与“短期套利后退出的经营者”。相关研究指出,质量不确定会诱发市场失灵。\footnote{George A. Akerlof, The Market for ``Lemons'': Quality Uncertainty and the Market Mechanism, \textit{The Quarterly Journal of Economics}, 1970, 84(3): 488--500.} 放到电商语境中,这种失灵往往表现为“低价—低质—难追责”的组合:当责任主体不清晰时,不诚信主体更容易通过价格战获得成交,而消费者在纠纷发生后才发现维权路径陡然变窄。

从消费者保护的角度看,信息公示至少具有三重功能。第一,它提供“识别功能”,让消费者知道交易相对人是谁,避免在商家名称、店铺名称、收款主体之间迷失。第二,它提供“证据功能”,消费者截图或保存的公示信息、规则信息与联系方式,常常是争议解决中最关键的证据之一。司法层面对电子数据的证据规则与互联网审判机制,实际上也在为“可留存、可核验”的信息披露提供制度支撑。\footnote{参见《最高人民法院关于互联网法院审理案件若干问题的规定》(法释〔2018〕16号),其中对电子数据、在线诉讼与平台取证的相关规则具有参考意义。} 第三,它提供“预防功能”,当经营者知道自身身份与资质处于可核验状态时,机会主义成本上升,守法经营的激励随之增强。类似的监管思路在欧盟平台治理框架中也有体现,例如通过增强在线交易主体透明度以减少消费者受骗风险。\footnote{参见欧盟《数字服务法案》(Digital Services Act, Regulation (EU) 2022/2065),其对平台透明度与商家可追责性提出系统性要求。}

因此,本文主张以“信息可得”作为解释轴心,将合规评价从“有没有贴出来”推进到“能不能被消费者有效使用”。所谓可得,不止是页面上存在信息,更在于信息能被找到、能被相信、能被用于救济:入口显著、路径短、跨端一致;内容真实、可核验、不过期;并且足以支撑联系、送达与追责。换言之,“亮照亮证”只是形式起点,“信息可得”才是效果终点。对电商治理而言,这是一项“底层工程”,看似朴素,却决定了很多更精细规则能否真正落地。



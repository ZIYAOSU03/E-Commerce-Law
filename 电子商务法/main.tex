\documentclass[12pt,a4paper]{article}

% 中文支持(格式参考:/Users/Zhuanz/Documents/哈吉馨/AI-Mediation-Paper-main/main.tex)
% 说明:为避免不同环境缺失 Windows 字体(如 SimSun/SimHei)导致编译失败,
% 这里使用 TeX Live 常见自带的 Fandol 字体集,保证可移植性。
\usepackage[UTF8,fontset=fandol]{ctex}

% 页面设置
\usepackage[top=2.5cm,bottom=2.5cm,left=3cm,right=2.5cm]{geometry}
\usepackage{setspace}
\onehalfspacing  % 1.5倍行距

% 数学和符号
\usepackage{amsmath}
\usepackage{amssymb}

% 图表
\usepackage{graphicx}
\usepackage{float}
\usepackage{caption}
\usepackage{subcaption}

% 超链接
\usepackage{hyperref}
\hypersetup{
    colorlinks=true,
    linkcolor=black,
    urlcolor=blue,
    citecolor=red
}

% 参考文献
\usepackage[style=gb7714-2015,backend=biber]{biblatex}
\addbibresource{references.bib}

% 其他包
\usepackage{enumitem}
\usepackage{titlesec}
\usepackage{fancyhdr}

% 页眉页脚
\pagestyle{fancy}
\fancyhf{}
\fancyhead[C]{从“亮照亮证”到“信息可得”——论电子商务经营者的信息公示义务及其实现路径}
\fancyfoot[C]{\thepage}
\renewcommand{\headrulewidth}{0.4pt}

% 标题格式
\titleformat{\section}{\Large\bfseries}{\thesection}{1em}{}
\titleformat{\subsection}{\large\bfseries}{\thesubsection}{1em}{}
\titleformat{\subsubsection}{\normalsize\bfseries}{\thesubsubsection}{1em}{}

% 标题信息(按需自行替换作者信息)
\title{从“亮照亮证”到“信息可得”——论电子商务经营者的信息公示义务及其实现路径}
\author{作者姓名}
\date{\today}

\begin{document}

% 封面
\maketitle
\thispagestyle{empty}
\newpage

% 目录
\tableofcontents
\newpage

% 摘要
\section*{摘要}
\addcontentsline{toc}{section}{摘要}

电子商务放大了交易中的信息不对称:消费者多依赖页面呈现,难以准确识别经营者身份与资质,纠纷中易出现“找不到人、追不了责”。为降低结构性风险,《电子商务法》确立经营者信息公示义务。本文主张将合规标准从“贴出证照”提升为“信息可得”,并围绕入口显著、内容真实、便于核验、变更同步提出改进建议,强调平台协同与责任闭环以提升制度实效。
电子商务放大了交易中的信息不对称:消费者往往在“看得见商品、看不清主体”的情境下完成下单,一旦发生质量争议或退款纠纷,才发现经营者联系方式不明、主体信息难以核验,维权成本随之上升。为降低这种结构性风险,《电子商务法》确立了经营者的信息公示义务,要求经营者在首页显著位置持续披露营业执照、许可资质与关键规则信息。本文结合常见交易场景,提出应当把合规标准从“贴出证照”提升为“信息可得”,并从可得性、可核验、可追溯三个维度阐释其制度价值与实践难点。文章进一步提出以统一入口、可点击核验、变更留痕与平台协同治理为抓手的改进路径,并辅以表格与示意图展示合规要点,力求使该义务真正转化为消费者可用的识别与救济能力。

\textbf{关键词:}电子商务法;信息公示义务;亮照亮证;消费者保护;平台治理




% 引言
\section{引言}

\subsection{选题背景与问题意识}
电子商务场景中,交易双方往往不见面、不签纸质合同、交易链条跨平台跨地域,消费者对经营者的了解高度依赖页面信息。与此同时,经营者“低成本进入—快速退出”的特征,使得虚假主体、资质不明、售后失联等风险更易发生。与传统线下交易相比,电子商务的效率优势在相当程度上来自信息与流程的数字化,但这一优势也可能被不诚信主体利用,形成对消费者不利的结构性信息差。

基于此,《中华人民共和国电子商务法》通过确立“经营者身份可识别”的基本要求,设置了以登记、公示、持续披露为核心的信息公示义务。该义务一方面为消费者提供最低限度的识别与追责线索,另一方面也为监管部门开展穿透式治理提供抓手。但在实践中,信息公示常被异化为“挂在角落的一张图片”或“看不懂、点不开的备案号”,出现展示不显著、信息不真实、变更不同步、跨端不一致、核验困难等问题,从而削弱了制度的实际效果。

\subsection{研究思路与结构安排}
本文选择“电子商务经营者信息公示义务”作为讨论对象,力图回答三个问题:第一,信息公示义务的规范内涵应如何理解,如何区分“形式合规”与“实质可得”;第二,该义务在消费者保护、市场秩序与监管效率上的制度价值何在;第三,针对实践中的落差,应当如何在经营者自律、平台协同与监管执法之间形成可执行的完善路径。

全文结构如下:第一章界定信息公示义务的规范基础与内涵边界;第二章从交易安全与治理逻辑角度阐释该义务的制度价值;第三章梳理实践痛点并分析成因;第四章提出以“可得性、可核验、可追溯”为导向的改进建议;结语部分总结观点并提出后续研究方向。




% 正文
\section{信息公示义务的规范要点与制度价值}

\subsection{规范要点:公示什么、如何公示、何时更新}
信息公示义务的核心不在“展示一张证照”,而在于确保经营者在网络交易中始终“可识别、可联系、可追责”。《中华人民共和国电子商务法》要求电子商务经营者依法登记,并在首页显著位置持续公示营业执照信息、行政许可信息等关键身份要素。\footnote{《中华人民共和国电子商务法》(2018年8月31日通过,2019年1月1日起施行)。}

从合规结构看,可以概括为“三个问题”:
\begin{enumerate}[label=(\arabic*)]
  \item \textbf{公示什么:}主体身份信息(名称、统一社会信用代码/登记信息、联系方式等)、许可资质信息(依法需许可的行业)、与消费者权益密切相关的规则信息(售后、争议处理等)。
  \item \textbf{如何公示:}入口易发现、少层级、跨端一致;表达应可理解,避免“贴图即完事”。
  \item \textbf{何时更新:}关键信息变更应及时同步,避免基于过期信息交易。
\end{enumerate}

\subsection{制度价值:把“信息差”转化为可管理的风险}
信息公示义务的重要性在于网络市场天然存在信息不对称:消费者难以区分“真实可追责”与“短期套利后退出”。相关研究指出,质量不确定会诱发市场失灵。\footnote{George A. Akerlof, The Market for ``Lemons'': Quality Uncertainty and the Market Mechanism, \textit{The Quarterly Journal of Economics}, 1970, 84(3): 488--500.} 信息公示通过降低识别与核验成本,提升交易确定性。

因此,本文主张以“信息可得”作为解释轴心,将合规评价从“有没有贴出来”推进到“能不能被消费者有效使用”。最低限度应满足:
\begin{enumerate}[label=(\arabic*)]
  \item \textbf{可见:}入口显著、少层级、跨端一致;
  \item \textbf{可信:}信息真实、可核验、不过期;
  \item \textbf{可用:}足以联系经营者、留存证据并支撑后续救济。
\end{enumerate}
换言之,“亮照亮证”只是形式起点,“信息可得”才是效果终点。



\section{信息公示义务的制度价值:消费者保护与市场治理的结合}

\subsection{降低信息不对称:把“信任成本”转化为可计算的风险}
网络交易中,消费者对经营者的了解主要来源于页面描述、评价体系与平台推荐机制。若缺乏身份与资质的最低披露,消费者只能在高度不确定中作出选择,信任成本上升,并最终表现为“劣币驱逐良币”:诚信经营者因合规成本与售后投入更高而处于劣势,不诚信经营者则通过低价与短周期获利后退出市场。

信息公示义务将经营者最关键的主体与资质信息制度化,使消费者能够在交易前进行基本风险判断,减少“买到假货才发现找不到人”的事后救济困境。其底层逻辑可用“信息不对称导致市场失灵”的经典观点来解释\cite{InformationAsymmetry1970}:当优质与劣质难以区分时,市场会系统性地惩罚诚信与质量。信息公示通过降低识别成本,帮助交易回到可竞争的轨道。这不仅服务于个体消费者,更能通过提升整体交易确定性,降低社会总交易成本。

\subsection{强化可追责性:为救济机制提供“抓手”}
消费者权益保护的难点常不在“权利是否存在”,而在“责任主体是否可确定”。在电子商务纠纷中,经营者失联、地址虚假、主体频繁变更等现象,会使投诉、调解、诉讼乃至强制执行陷入空转。信息公示义务通过“经营者身份可识别”这一底层设计,为后续的合同责任、侵权责任与行政责任衔接提供基础要素。

进一步而言,信息公示还具有证据功能:消费者截图留存的主体信息、规则信息与联系方式,在争议解决中往往构成关键证据。若经营者被允许以“隐藏信息”方式经营,则实际上变相剥夺了消费者的证据准备能力。

\subsection{提升监管效率:从事后处罚走向事前预防}
对监管机关而言,互联网交易规模巨大,单纯依赖事后处罚难以覆盖全量风险。信息公示义务为监管提供了“低成本筛查”的入口:通过公示信息的完整性、真实性与更新及时性,可以对高风险主体开展重点监管;通过平台侧的技术核验与黑名单共享,可以实现早发现、早处置。

因此,信息公示义务兼具私法与公法双重价值:在私法层面,它保护交易相对方的知情与选择;在公法层面,它服务于市场秩序与监管可达性。其关键不在“是否张贴”,而在“是否能被利用来形成有效治理”。



\section{实践痛点与成因分析:形式合规为何难以转化为实质效果}

\subsection{痛点一:入口不显著与跨端不一致}
不少经营者在网页端展示相对完整,但在移动端入口被折叠在多级菜单之下;或者在小程序、直播间、短链页面中难以找到主体信息。对消费者而言,“找不到”在效果上等同于“未公示”。跨端不一致还会引发证据与责任认定困难:消费者截图显示的信息与经营者后台登记信息不一致,争议解决成本随之上升。

\subsection{痛点二:真实性与可核验不足}
即便展示了营业执照图片,也可能存在过期、伪造、裁剪关键信息等问题;行政许可信息亦可能未同步更新。消费者缺乏便捷核验渠道,监管机关也难以对海量主体逐一人工核查。结果是公示信息沦为“装饰性合规”,无法发挥筛选与预防功能。

\subsection{痛点三:变更不同步与“动态逃逸”}
电子商务经营具有高流动性,经营者可能频繁变更名称、地址、主体或实际控制人;若变更后未及时公示更新,消费者将面临“追责对象错位”。更极端的情形是利用多主体轮换、店铺转让、关联主体拆分等方式进行“动态逃逸”,使得个案责任追究与失信惩戒难以落地。

\subsection{痛点四:信息过载与可理解性不足}
部分平台或经营者为规避风险,将大量格式条款与规则以冗长文本一次性堆叠展示,消费者难以抓取关键内容;或者用专业术语、跳转链接规避显著提示义务。信息公示若只追求“数量”而忽视“可理解”,同样无法实现知情与选择。

\subsection{成因分析:激励失衡、平台责任边界与执法成本}
上述问题的成因具有结构性:
\begin{enumerate}[label=(\arabic*)]
  \item \textbf{经营者激励失衡:}充分披露会增加被投诉与被追责概率,不诚信主体有动机隐藏信息或延迟更新。
  \item \textbf{平台治理差异:}平台既掌握流量入口与技术能力,也承受合规成本与用户体验压力;若缺乏统一标准与强约束,治理会出现“选择性投入”。
  \item \textbf{执法资源约束:}主体数量巨大,若仅依赖行政机关线下核查,将导致发现率低、处罚滞后,难以形成稳定预期。近年来针对网络交易的专门规制也不断细化,对公示、核验、平台义务等提出更明确要求\cite{SAMRECommerce2021},但在具体执行层面仍需要技术与流程支撑。
\end{enumerate}

因此,完善信息公示义务不能止于“再强调一次要公示”,而应通过可执行标准、技术核验与责任联动,把形式义务转化为可验证、可追责、可持续的治理机制。



\section{完善路径:以“可得性、可核验、可追溯”为中心的制度组合}

\subsection{确立可执行的分层公示标准}
建议将信息公示分为“基础必备信息”与“特定业态补充信息”两层:基础层统一要求经营者在所有端口(网页、APP、小程序、直播间/带货入口)提供一键可达的主体身份与联系方式;补充层则针对食品药品、教育培训、跨境电商等高风险领域增加许可、资质、投诉渠道等要素。通过分层标准既能避免“一刀切”造成合规负担过重,也能提升高风险领域的消费者保护强度。

\subsection{引入强制可核验机制:从“贴图”走向“可校验标识”}
仅展示图片难以防伪。可借鉴“可点击校验”的思路:平台侧在经营者完成资质提交与核验后,生成统一的可核验标识(如可点击跳转到平台核验页,展示核验时间、有效期、许可范围等)。监管部门可推动平台与登记系统/许可系统对接,实现自动比对与到期提醒。这样既降低消费者核验成本,也使平台治理与行政监管形成数据联动。

\subsection{强化动态更新与变更提示:让信息披露贯穿交易全流程}
建议对信息变更设置更明确的“同步义务”与“提示义务”:经营者变更名称、地址、许可范围等关键信息时,除更新公示页外,应在一定期间内对已下单消费者提供可见提示(如订单页提示、站内信提示),并确保历史订单可追溯到当时有效信息。对频繁变更且投诉率高的主体,平台可实施重点审查或限制经营措施。

\subsection{平台协同治理:把入口责任转化为可监督的流程责任}
平台掌握流量入口与交易链路,是实现“信息可得”的关键节点。建议在平台规则中明确:未完成基础信息公示与核验的经营者不得开店或不得上架商品;对直播带货等新业态,应确保主体信息在直播间界面持续可见。同时引入外部可审计机制,例如定期披露平台抽检比例、违规处置数据,并接受监管抽查或第三方评估,避免平台治理沦为“内部口径”。

\subsection{责任与惩戒衔接:让违法成本高于逃逸收益}
对拒不公示、虚假公示、长期不更新等行为,应在行政处罚、信用惩戒与平台处置之间形成合力:行政机关依法处罚并纳入信用记录;平台实施限流、下架、关店等处置;对造成严重后果的,可依法追究民事赔偿乃至刑事责任。通过提高综合违法成本,改变不诚信主体的成本收益结构,使信息公示从“可做可不做”变为“必须做好”。




% 结论
\section{结语}
信息公示义务的关键,不是“有没有展示”,而是“能不能被有效使用”。如果入口隐蔽、信息不可核验、变更不同步,消费者仍会陷入识别与追责困境。本文以“信息可得”为解释轴心,提出以可得性(显著入口)、可核验(点击校验)、可追溯(历史留痕)为核心的改进方向,并强调平台协同与责任闭环。通过把规则转化为可执行流程,才能让“亮照亮证”真正服务于交易安全与消费者保护。

同时也应看到,信息披露并非越多越好:过度披露可能造成信息过载,反而削弱消费者的理解与选择。未来完善制度时,应在“关键事实充分披露”与“用户可理解”之间取得平衡,并结合不同业态风险水平动态调整披露与核验强度,使规则既能落地又不损害交易效率。

回到课堂讨论的语境,我对信息公示义务的理解更偏向“基础设施”:它看似不如算法治理、平台责任分配那样“宏大”,却决定了消费者能否以较低成本进入救济通道,也决定了平台治理与行政监管能否找到抓手。若能把统一入口、可核验标识、变更留痕这些机制长期稳定地运行起来,很多纠纷会在早期就被止损;反之,即使规则写得再完整,最终也可能因为“找不到人”而落空。这种从形式展示走向实质可得的转变,正是电子商务法治化在细节处的成败关键。




% 参考文献
\newpage
\printbibliography[title=参考文献]

\end{document}



\documentclass[12pt,a4paper]{article}

% ==================================================
% 1. 基础宏包与字体设置
% ==================================================
\usepackage[UTF8, scheme=chinese, fontset=fandol]{ctex} % 开启中文支持与可移植字体
\usepackage{fontspec}
% 如需指定系统字体可自行打开:
% \setmainfont{SimSun}
% \setsansfont{SimHei}

% ==================================================
% 2. 页面设置
% ==================================================
\usepackage[top=2.5cm,bottom=2.5cm,left=3cm,right=2.5cm]{geometry}
\usepackage{setspace}
\onehalfspacing  % 1.5倍行距

% ==================================================
% 3. 数学、图表与工具包
% ==================================================
\usepackage{amsmath}
\usepackage{amssymb}
\usepackage{graphicx}
\usepackage{float}
\usepackage{caption}
\usepackage{subcaption}
\usepackage{enumitem}
\usepackage{titlesec}
\usepackage{fancyhdr}
\usepackage{tikz}
\usetikzlibrary{shapes.geometric, arrows.meta, positioning, shadows}
\usepackage{tabularx}
\usepackage{booktabs}
\usepackage{array}
% ==================================================
% 标题格式设置 (统一为 1 / 1.1 格式)
% ==================================================
\usepackage{titlesec}

% --- 重定义编号计数器为阿拉伯数字 ---
% 这将确保目录和正文同步为 1, 1.1, 1.1.1
\renewcommand{\thesection}{\arabic{section}}
\renewcommand{\thesubsection}{\thesection.\arabic{subsection}}
\renewcommand{\thesubsubsection}{\thesubsection.\arabic{subsubsection}}

% --- 设置正文标题样式 ---
% 一级标题:如 "1 引言"
\titleformat{\section}{\Large\bfseries}{\thesection}{0.5em}{}

% 二级标题:如 "1.1 研究背景"
\titleformat{\subsection}{\large\bfseries}{\thesubsection}{0.5em}{}

% 三级标题:如 "1.1.1 细节说明"
\titleformat{\subsubsection}{\normalsize\bfseries}{\thesubsubsection}{0.5em}{}

% ==================================================
% 5. 超链接与参考文献
% ==================================================
\usepackage{hyperref}
\hypersetup{
    colorlinks=true,
    linkcolor=black,
    urlcolor=blue,
    citecolor=red,
    bookmarksnumbered=true % 书签中显示章节编号
}

% 本文引用采用脚注形式,不使用 biblatex(避免 biber 依赖)

% ==================================================
% 6. 页眉页脚
% ==================================================
\pagestyle{fancy}
\fancyhf{}
\fancyhead[C]{从“亮照亮证”到“信息可得”——电子商务经营者信息公示义务研究}
\fancyfoot[C]{\thepage}
\setlength{\headheight}{15pt}
\renewcommand{\headrulewidth}{0.4pt}

% ==================================================
% 7. 文档信息
% ==================================================
\title{从“亮照亮证”到“信息可得”——论电子商务经营者的信息公示义务及其实现路径}
\author{张馨}
\date{\today}

% ==================================================
% 8. 正文开始
% ==================================================
\begin{document}

% ==================================================
% 优化后的封面 (titlepage)
% ==================================================
\begin{titlepage}
    \centering % 全页居中

    % 1. 校徽位置
    % 说明:请将你的校徽图片命名为 logo.png 并放在项目文件夹中
    % 如果还没有图片,目前会显示一个占位用的黑框
    \vspace*{1cm} % 距离顶部的距离
    \begin{figure}[h]
        \centering
        % -----------------------------------------------------------
        % 替换下方文件名 'example-image' 为你的校徽文件名,如 'logo.png'
        % 调整 width=... 可以改变图片大小
        % -----------------------------------------------------------
        \IfFileExists{bupt-log.png}{%
            \includegraphics[width=12cm]{bupt-log.png}%
        }{%
            \rule{12cm}{3cm}%
        }
    \end{figure}
    
    \vspace{0cm} % 校徽和标题之间的距离

% 2. 论文标题
    { \Huge \bfseries 从“亮照亮证”到“信息可得”——\\[0.5em] 论电子商务经营者的信息公示义务及其实现路径 \par}
    
    \vspace{3cm} % 标题和个人信息之间的距离

    % 3. 个人信息 (使用表格对齐,显得更专业)
    {\Large
    \renewcommand{\arraystretch}{1.6} % 表格行高
    \begin{tabular}{r@{\hspace{1em}}l} % r=右对齐标签,l=左对齐内容
        \textbf{论文作者:} & 张馨 \\
        \textbf{学\hspace{2em}号:} & 2023213104 \\
        \textbf{学\hspace{2em}院:} & 人文学院 \\
        \textbf{专\hspace{2em}业:} & 法学 \\
    \end{tabular}
    }

    \vfill % 核心命令:自动填充垂直空间,将下方的日期推到页面最底部

    % 4. 底部日期
    {\Large \today}
    
    \vspace*{2cm} % 距离底部的留白
\end{titlepage}

% 这里的命令确保封面不编页码,且目录从第 I 页或第 1 页开始
\thispagestyle{empty} 
\newpage

% ------ 目录 ------
\tableofcontents
\newpage

% ------ 摘要 ------
\section*{摘要}
\addcontentsline{toc}{section}{摘要}

电子商务放大了交易中的信息不对称:消费者多依赖页面呈现,难以准确识别经营者身份与资质,纠纷中易出现“找不到人、追不了责”。为降低结构性风险,《电子商务法》确立经营者信息公示义务。本文主张将合规标准从“贴出证照”提升为“信息可得”,并围绕入口显著、内容真实、便于核验、变更同步提出改进建议,强调平台协同与责任闭环以提升制度实效。
电子商务放大了交易中的信息不对称:消费者往往在“看得见商品、看不清主体”的情境下完成下单,一旦发生质量争议或退款纠纷,才发现经营者联系方式不明、主体信息难以核验,维权成本随之上升。为降低这种结构性风险,《电子商务法》确立了经营者的信息公示义务,要求经营者在首页显著位置持续披露营业执照、许可资质与关键规则信息。本文结合常见交易场景,提出应当把合规标准从“贴出证照”提升为“信息可得”,并从可得性、可核验、可追溯三个维度阐释其制度价值与实践难点。文章进一步提出以统一入口、可点击核验、变更留痕与平台协同治理为抓手的改进路径,并辅以表格与示意图展示合规要点,力求使该义务真正转化为消费者可用的识别与救济能力。

\textbf{关键词:}电子商务法;信息公示义务;亮照亮证;消费者保护;平台治理




% ------ 正文(两个章节) ------
\section{引言}

\subsection{选题背景与问题意识}
电子商务场景中,交易双方往往不见面、不签纸质合同、交易链条跨平台跨地域,消费者对经营者的了解高度依赖页面信息。与此同时,经营者“低成本进入—快速退出”的特征,使得虚假主体、资质不明、售后失联等风险更易发生。与传统线下交易相比,电子商务的效率优势在相当程度上来自信息与流程的数字化,但这一优势也可能被不诚信主体利用,形成对消费者不利的结构性信息差。

基于此,《中华人民共和国电子商务法》通过确立“经营者身份可识别”的基本要求,设置了以登记、公示、持续披露为核心的信息公示义务。该义务一方面为消费者提供最低限度的识别与追责线索,另一方面也为监管部门开展穿透式治理提供抓手。但在实践中,信息公示常被异化为“挂在角落的一张图片”或“看不懂、点不开的备案号”,出现展示不显著、信息不真实、变更不同步、跨端不一致、核验困难等问题,从而削弱了制度的实际效果。

\subsection{研究思路与结构安排}
本文选择“电子商务经营者信息公示义务”作为讨论对象,力图回答三个问题:第一,信息公示义务的规范内涵应如何理解,如何区分“形式合规”与“实质可得”;第二,该义务在消费者保护、市场秩序与监管效率上的制度价值何在;第三,针对实践中的落差,应当如何在经营者自律、平台协同与监管执法之间形成可执行的完善路径。

全文结构如下:第一章界定信息公示义务的规范基础与内涵边界;第二章从交易安全与治理逻辑角度阐释该义务的制度价值;第三章梳理实践痛点并分析成因;第四章提出以“可得性、可核验、可追溯”为导向的改进建议;结语部分总结观点并提出后续研究方向。



\section{信息公示义务的规范要点与制度价值}

\subsection{规范要点:公示什么、如何公示、何时更新}
信息公示义务的核心不在“展示一张证照”,而在于确保经营者在网络交易中始终“可识别、可联系、可追责”。《中华人民共和国电子商务法》要求电子商务经营者依法登记,并在首页显著位置持续公示营业执照信息、行政许可信息等关键身份要素。\footnote{《中华人民共和国电子商务法》(2018年8月31日通过,2019年1月1日起施行)。}

从合规结构看,可以概括为“三个问题”:
\begin{enumerate}[label=(\arabic*)]
  \item \textbf{公示什么:}主体身份信息(名称、统一社会信用代码/登记信息、联系方式等)、许可资质信息(依法需许可的行业)、与消费者权益密切相关的规则信息(售后、争议处理等)。
  \item \textbf{如何公示:}入口易发现、少层级、跨端一致;表达应可理解,避免“贴图即完事”。
  \item \textbf{何时更新:}关键信息变更应及时同步,避免基于过期信息交易。
\end{enumerate}

\subsection{制度价值:把“信息差”转化为可管理的风险}
信息公示义务的重要性在于网络市场天然存在信息不对称:消费者难以区分“真实可追责”与“短期套利后退出”。相关研究指出,质量不确定会诱发市场失灵。\footnote{George A. Akerlof, The Market for ``Lemons'': Quality Uncertainty and the Market Mechanism, \textit{The Quarterly Journal of Economics}, 1970, 84(3): 488--500.} 信息公示通过降低识别与核验成本,提升交易确定性。

因此,本文主张以“信息可得”作为解释轴心,将合规评价从“有没有贴出来”推进到“能不能被消费者有效使用”。最低限度应满足:
\begin{enumerate}[label=(\arabic*)]
  \item \textbf{可见:}入口显著、少层级、跨端一致;
  \item \textbf{可信:}信息真实、可核验、不过期;
  \item \textbf{可用:}足以联系经营者、留存证据并支撑后续救济。
\end{enumerate}
换言之,“亮照亮证”只是形式起点,“信息可得”才是效果终点。



\section{信息公示义务的制度价值:消费者保护与市场治理的结合}

\subsection{降低信息不对称:把“信任成本”转化为可计算的风险}
网络交易中,消费者对经营者的了解主要来源于页面描述、评价体系与平台推荐机制。若缺乏身份与资质的最低披露,消费者只能在高度不确定中作出选择,信任成本上升,并最终表现为“劣币驱逐良币”:诚信经营者因合规成本与售后投入更高而处于劣势,不诚信经营者则通过低价与短周期获利后退出市场。

信息公示义务将经营者最关键的主体与资质信息制度化,使消费者能够在交易前进行基本风险判断,减少“买到假货才发现找不到人”的事后救济困境。其底层逻辑可用“信息不对称导致市场失灵”的经典观点来解释\cite{InformationAsymmetry1970}:当优质与劣质难以区分时,市场会系统性地惩罚诚信与质量。信息公示通过降低识别成本,帮助交易回到可竞争的轨道。这不仅服务于个体消费者,更能通过提升整体交易确定性,降低社会总交易成本。

\subsection{强化可追责性:为救济机制提供“抓手”}
消费者权益保护的难点常不在“权利是否存在”,而在“责任主体是否可确定”。在电子商务纠纷中,经营者失联、地址虚假、主体频繁变更等现象,会使投诉、调解、诉讼乃至强制执行陷入空转。信息公示义务通过“经营者身份可识别”这一底层设计,为后续的合同责任、侵权责任与行政责任衔接提供基础要素。

进一步而言,信息公示还具有证据功能:消费者截图留存的主体信息、规则信息与联系方式,在争议解决中往往构成关键证据。若经营者被允许以“隐藏信息”方式经营,则实际上变相剥夺了消费者的证据准备能力。

\subsection{提升监管效率:从事后处罚走向事前预防}
对监管机关而言,互联网交易规模巨大,单纯依赖事后处罚难以覆盖全量风险。信息公示义务为监管提供了“低成本筛查”的入口:通过公示信息的完整性、真实性与更新及时性,可以对高风险主体开展重点监管;通过平台侧的技术核验与黑名单共享,可以实现早发现、早处置。

因此,信息公示义务兼具私法与公法双重价值:在私法层面,它保护交易相对方的知情与选择;在公法层面,它服务于市场秩序与监管可达性。其关键不在“是否张贴”,而在“是否能被利用来形成有效治理”。




% ------ 结论 ------
% 第八部分:结论
\section{结语}
信息公示义务的关键,不是“有没有展示”,而是“能不能被有效使用”。如果入口隐蔽、信息不可核验、变更不同步,消费者仍会陷入识别与追责困境。本文以“信息可得”为解释轴心,提出以可得性(显著入口)、可核验(点击校验)、可追溯(历史留痕)为核心的改进方向,并强调平台协同与责任闭环。通过把规则转化为可执行流程,才能让“亮照亮证”真正服务于交易安全与消费者保护。

同时也应看到,信息披露并非越多越好:过度披露可能造成信息过载,反而削弱消费者的理解与选择。未来完善制度时,应在“关键事实充分披露”与“用户可理解”之间取得平衡,并结合不同业态风险水平动态调整披露与核验强度,使规则既能落地又不损害交易效率。

回到课堂讨论的语境,我对信息公示义务的理解更偏向“基础设施”:它看似不如算法治理、平台责任分配那样“宏大”,却决定了消费者能否以较低成本进入救济通道,也决定了平台治理与行政监管能否找到抓手。若能把统一入口、可核验标识、变更留痕这些机制长期稳定地运行起来,很多纠纷会在早期就被止损;反之,即使规则写得再完整,最终也可能因为“找不到人”而落空。这种从形式展示走向实质可得的转变,正是电子商务法治化在细节处的成败关键。




% 本文引用采用脚注形式,如需另附参考文献表,可在此处补充“参考资料”章节。

% ------ 附录(可选) ------
% \newpage
% \appendix
% \input{chapters/appendix}

\end{document}